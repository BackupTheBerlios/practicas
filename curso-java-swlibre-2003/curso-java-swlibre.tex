\documentclass[notes,a4,slidesec,dvips]{seminar}
\usepackage[isolatin]{inputenc}
\usepackage{fancybox}
\usepackage{epsfig}
\usepackage{moreverb}
\usepackage{slnuevo}
\usepackage[spanish,activeacute]{babel}
\usepackage{graphicx}
\def\contstr{(contd.)}

\title{Java y Software Libre}
\author{\parbox{5.2cm}{\center javaHispano\\}}

\date{4 de abril de 2003}
\cop{2002 javaHispano}
%\address{\ }
\conference{�lvaro S�nchez-Mariscal (\texttt{mariscal@javahispano.org}) \\
			Alberto Molpeceres (\texttt{al@javahispano.org}) \\
			Mart�n P�rez (\texttt{martin@javahispano.org})}

\sltitle{Java y Software Libre}


\begin{document}
\selectlanguage{spanish}
\maketitle 

%% -------------------------------------------------------------- 
\begin{hslide}
\slsect{Contenido}
	\begin{itemize}
	\item Evoluci�n de Java: desde 1995 a la actualidad.
	\item El JCP y las especificaciones.
	\item Algunos datos estad�sticos sobre el JCP.
	\item Implementaciones libres.
	\item Dudas y preguntas.
	\end{itemize}
\end{hslide}  
%% -------------------------------------------------------------- 
%% -------------------------------------------------------------- 
\begin{hslide}
\slsect{Java, un nuevo lenguaje}
	\begin{itemize}
	\item Nace en 1995.
	\item Lenguaje de programaci�n propietario de SUN.
		  \begin{itemize}
		  \item Sin embargo, SUN era consciente de que era necesario
				implicar a cuantas m�s empresas y desarrolladores
				fuese posible. 
		  \end{itemize}
	\item Problemas:
		  \begin{itemize}
		  \item La propiedad del software ser� siempre de SUN y/o sus
				licenciatarios.
		  \item No se permit�an implementaciones libres de las librer�as
		    	(\texttt{java.*}, \texttt{javax.*} y \texttt{sun.*}).
	      \item Se prohib�a distribuir el JDK o el JRE de SUN desde sitios
		  		externos a la propia SUN.
		  \end{itemize}

	\newpage

	\item Aparecen ya grupos de software libre, como \textit{Kaffe} y
		  \textit{Japhar}.
	\item Esta situaci�n era muy negativa para el mundo del software
     	  libre. 
		  \begin{itemize}
		  \item Sin embargo, mucha gente estaba de acuerdo con esta
			    situaci�n.
			    \begin{itemize}
			    \item As� se favorece un desarrollo uniforme y se
					  protege la plataforma de versiones devaluadas
					  (kit de desarrollo de Microsoft).
			    \end{itemize}
		  \end{itemize}
	\end{itemize}

\end{hslide}  
%% -------------------------------------------------------------- 
%% -------------------------------------------------------------- 
\begin{hslide}
\slsect{La creaci�n del JCP}
	\begin{itemize}
	\item Poco a poco Java se iba extendiendo cada vez m�s.
	\item Comenzaban a surgir numerosas tecnolog�as: EJB,
		  Servlets/JSP, etc.
	\item Surgi� la necesidad de estandarizar las especificaciones.
		  \begin{itemize}
		  \item Hac�a falta que un organismo imparcial controlara los
		        distintos desarrollos.
		  \end{itemize}
	\item Primera opci�n: la ISO.
		  \begin{itemize}
		  \item Descartada. SUN dec�a que la excesiva burocracia de
				estos organismos acabar�a con el dinamismo de Java.
		  \item Ejemplo: CORBA.
		  \item Esta decisi�n provoc� muchas cr�ticas.
		  \end{itemize}

	\newpage

	\item Por todo lo anterior, SUN decide crear el \textbf{Java 
		  Community Process} (JCP) en diciembre de 1998.
	\item Organismo encargado de asegurar la evoluci�n de Java.
	\item Controla la creaci�n, evoluci�n y aprobaci�n de las
		  especificaciones.

	\end{itemize}

\end{hslide}  
%% -------------------------------------------------------------- 



%% -------------------------------------------------------------- 
\begin{hslide}
\slsect{}
\end{hslide}  
%% -------------------------------------------------------------- 


\end{document}

